
\begin{DoxyCode}
pembuat : Rifki Sadikin (rifki.sadikin@lipi.go.id)
tanggal : 6 November 2018
\end{DoxyCode}


Petunjuk penggunaan ini berisi\+:


\begin{DoxyItemize}
\item Cara {\bfseries membuat} dan {\bfseries menginstal} Pustaka Pemercepat Komputasi berbasis G\+PU C\+U\+DA untuk Penyelesaian Persamaan {\itshape Poisson} dalam Koordinat Silindrikal.
\item Cara {\bfseries memakai} Pustaka Pemercepat Komputasi berbasis G\+PU C\+U\+DA untuk Penyelesaian Persamaan {\itshape Poisson} dalam Koordinat Silindrikal dalam C++.
\end{DoxyItemize}

\subsection*{Struktur Berkas}

Struktur berkas pustaka Pemercepat Komputasi berbasis G\+PU C\+U\+DA untuk Penyelesaian Persamaan {\itshape Poisson} dalam Koordinat Silindrikal adalah sebagai berikut\+:


\begin{DoxyCode}
|-- CMakeLists.txt
|-- docs
|   `-- Doxyfile.in
|-- example
|   |-- CMakeLists.txt
|   |-- PoissonSolver3DGPUTest.cpp
|   `-- PoissonSolver3DGPUTest.h
|-- interface
|   |-- PoissonSolver3DCylindricalGPU.cxx
|   `-- PoissonSolver3DCylindricalGPU.h
|-- kernel
|   |-- PoissonSolver3DGPU.cu
|   `-- PoissonSolver3DGPU.h
|-- PETUNJUKPENGGUNAAN.md
`-- README.md
\end{DoxyCode}
 \subsection*{Modul Program yang Dibutuhkan}

Untuk menginstall pustaka dibutuhkan program pengembang di lingkungan sistem operasi Linux (saat ini masih di Linux) yaitu\+:


\begin{DoxyEnumerate}
\item C\+M\+A\+KE versi minimal 3.\+10.\+3
\item G\+CC versi minimal 6.\+2.\+0
\item C\+U\+DA versi minimal 8.\+0
\item G\+IT versi minimal 1.\+8.\+3.\+1
\end{DoxyEnumerate}

Jika disediakan module program pada lingkungan sistem H\+PC maka dapat menjalankan perintah berikut ini\+:


\begin{DoxyCode}
module load cmake/3.10.3
module load gcc/6.2.0
module load cuda
\end{DoxyCode}
 \subsection*{Instalasi}

Untuk menginstall pustaka Poisson\+Solver3D lakukan langkah-\/langkah berikut\+:


\begin{DoxyEnumerate}
\item Jalankan git clone untuk menyalin kode sumber ke direktori local. Perintah ini akan menyalin kode sumber ke direktori {\bfseries Poisson\+Solver3D} 
\begin{DoxyCode}
$ git clone http://github.com/rsadikin/PoissonSolver3D.git
\end{DoxyCode}

\item Pindah ke direktori Poisson\+Solver3D, dan lihat daftar direktori 
\begin{DoxyCode}
$ cd PoissonSolver3D
$ tree .
.
|-- CMakeLists.txt
|-- docs
|   `-- Doxyfile.in
|-- example
|   |-- CMakeLists.txt
|   |-- PoissonSolver3DGPUTest.cpp
|   `-- PoissonSolver3DGPUTest.h
|-- interface
|   |-- PoissonSolver3DCylindricalGPU.cxx
|   `-- PoissonSolver3DCylindricalGPU.h
|-- kernel
|   |-- PoissonSolver3DGPU.cu
|   `-- PoissonSolver3DGPU.h
|-- PETUNJUKPENGGUNAAN.md
`-- README.md
\end{DoxyCode}

\item Buatlah folder {\bfseries buildpoissonsolver} di luar direktori kode sumber 
\begin{DoxyCode}
$ mkdir $HOME/buildpoissonsolver
\end{DoxyCode}

\item Pindah ke folder {\bfseries buildpoissonsolver}
\end{DoxyEnumerate}


\begin{DoxyCode}
$ cd $HOME/buildpoissonsolver
\end{DoxyCode}

\begin{DoxyEnumerate}
\item Jalankan {\bfseries cmake} di folder {\bfseries buildpoissonsolver} dengan menyatakan di mana pustaka akan diinstal dengan perintah sebagai berikut 
\begin{DoxyCode}
$ export  PSLIB=/home/usertest/trypoissonsolver/PoissonSolver3D
$ export PSLIB\_INSTALL=/home/usertest/buildpoissonsolver
$ cmake $PSLIB/ -DCMAKE\_INSTALL\_PREFIX=$PSLIB\_INSTALL
-- The C compiler identification is GNU 6.2.0
-- The CXX compiler identification is GNU 6.2.0
-- Check for working C compiler: /apps/tools/gcc-6.2.0/bin/gcc
-- Check for working C compiler: /apps/tools/gcc-6.2.0/bin/gcc -- works
-- Detecting C compiler ABI info
-- Detecting C compiler ABI info - done
-- Detecting C compile features
-- Detecting C compile features - done
-- Check for working CXX compiler: /apps/tools/gcc-6.2.0/bin/c++
-- Check for working CXX compiler: /apps/tools/gcc-6.2.0/bin/c++ -- works
-- Detecting CXX compiler ABI info
-- Detecting CXX compiler ABI info - done
-- Detecting CXX compile features
-- Detecting CXX compile features - done
-- Looking for pthread.h
-- Looking for pthread.h - found
-- Looking for pthread\_create
-- Looking for pthread\_create - not found
-- Looking for pthread\_create in pthreads
-- Looking for pthread\_create in pthreads - not found
-- Looking for pthread\_create in pthread
-- Looking for pthread\_create in pthread - found
-- Found Threads: TRUE
-- Found CUDA: /apps/tools/cuda-9.1 (found version "9.1")
-- Building SpaceCharge distortion framework with CUDA support
-- Found Doxygen: /usr/bin/doxygen (found version "1.8.5") found components:  doxygen dot
Doxygen build started
-- Configuring done
-- Generating done
-- Build files have been written to: /home/usertest/buildpoissonsolver
\end{DoxyCode}

\item Kompail kode sumbernya 
\begin{DoxyCode}
$ make
\end{DoxyCode}

\item Install modul Poisson\+Solver3D di folder {\bfseries buildpoissonsolver} 
\begin{DoxyCode}
$ make install
..
-- Install configuration: ""
-- Installing: /home/usertest/trypoissonsolver/buildpoissonsolver/lib/libPoissonSolver3DCylindricalGPU.so
-- Installing: /home/usertest/trypoissonsolver/buildpoissonsolver/include/PoissonSolver3DCylindricalGPU.h
-- Installing: /home/usertest/trypoissonsolver/buildpoissonsolver/include/PoissonSolver3DGPU.h
\end{DoxyCode}

\end{DoxyEnumerate}

Hasil instalasi adalah sebuah pustaka yang dapat digunakan ({\bfseries shared library}) yaitu lib\+Poisson\+Solver3\+D\+Cylindrical\+G\+P\+U.\+so pada direktori {\bfseries lib} dan 2 berkas header yang mengandung definisi kelas /fungsi sehingga pengguna pustaka dapat menggunakannya.

\subsection*{Penggunaan Pustaka}

\subsubsection*{Penggunaan Dalam C\+M\+A\+KE}

Dalam folder {\bfseries example} di struktur direktori sumber terdapat contoh penggunaan pustaka libpoissonsolvergpu.\+so dengan menggunakan {\bfseries cmake}. Berikut ini langkah-\/langkahnya\+:


\begin{DoxyEnumerate}
\item Buat file {\bfseries C\+Make\+Lists.\+txt} dengan struktur sebagai berikut untuk menambahkan pustaka cuda ({\bfseries libcuda.\+so}) dan pustaka \hyperlink{classPoissonSolver3DCylindricalGPU}{Poisson\+Solver3\+D\+Cylindrical\+G\+PU} ({\bfseries lib\+Poisson\+Solver3\+D\+Cylindrical\+G\+P\+U.\+so}) pada proyek cmake\+:
\end{DoxyEnumerate}


\begin{DoxyCode}
cmake\_minimum\_required (VERSION 2.8.11)
project (3DPoissonSolverGPUTest)

find\_package(CUDA)
if(NOT CUDA\_FOUND)
    message( FATAL\_ERROR "NVIDIA CUDA package not found" )
else()
    find\_library(LIBCUDA\_SO\_PATH libcuda.so)
    string(FIND $\{LIBCUDA\_SO\_PATH\} "-NOTFOUND" LIBCUDA\_SO\_PATH\_NOTFOUND )
endif(NOT CUDA\_FOUND)
message( STATUS "Building Poisson Solver  with CUDA support" )

if(LIBCUDA\_SO\_PATH\_NOTFOUND GREATER -1)
  message( FATAL\_ERROR "NVIDIA CUDA libcuda.so not found" )
endif(LIBCUDA\_SO\_PATH\_NOTFOUND GREATER -1)


#set(CMAKE\_MODULE\_PATH "$\{CMAKE\_SOURCE\_DIR\}/CMake/cuda" $\{CMAKE\_MODULE\_PATH\})
#find\_package(CUDA QUIET REQUIRED)
set(PSLIBNAME libPoissonSolver3DCylindricalGPU.so)

find\_library(PSLIB $\{PSLIBNAME\})
string(FIND $\{PSLIB\} "-NOTFOUND" PSLIB\_NOTFOUND )

if(PSLIB\_NOTFOUND GREATER -1)
  message( FATAL\_ERROR "Poisson Solver Cuda Library libPoissonSolver3DCylindricalGPU.o not found" )
endif(PSLIB\_NOTFOUND GREATER -1)
\end{DoxyCode}
 Setelah itu, baru tambahkan perintah pada C\+Make\+Lists.\+txt (dilanjutkan) untuk mengikut sertakan kode sumber user yaitu\+:


\begin{DoxyCode}
# tambah disini kode sumber user
set(CPP\_SOURCE PoissonSolver3DGPUTest.cpp)
set(HEADERS PoissonSolver3DGPUTest.h)

set(TARGET\_NAME poissonsolvergputest)

add\_executable($\{TARGET\_NAME\}
  PoissonSolver3DGPUTest.cpp
)


# ikut sertakan shared library cuda dan poisson solver
target\_link\_libraries($\{TARGET\_NAME\} $\{PSLIB\} $\{LIBCUDA\_SO\_PATH\})
\end{DoxyCode}



\begin{DoxyEnumerate}
\item Pada kode sumber include header file sehingga definisi fungsi dan kelas dapat dipanggil di badan kode.
\end{DoxyEnumerate}


\begin{DoxyCode}
\{c++\}
#include "PoissonSolver3DCylindricalGPU.h"

...
// create poissonSolver
PoissonSolver3DCylindricalGPU *poissonSolver = new PoissonSolver3DCylindricalGPU();
PoissonSolver3DCylindricalGPU::fgConvergenceError = 1e-8;
poissonSolver->SetExactSolution(VPotentialExact,kRows,kColumns, kPhiSlices);
poissonSolver->SetStrategy(PoissonSolver3DCylindricalGPU::kMultiGrid);
poissonSolver->SetCycleType(PoissonSolver3DCylindricalGPU::kFCycle);
poissonSolver->PoissonSolver3D(VPotential,RhoCharge,kRows,kColumns,kPhiSlices, kIterations,kSymmetry) ;
\end{DoxyCode}



\begin{DoxyEnumerate}
\item Buat folder terpisah untuk mebangun proyek yang menggunakan pustaka misal {\bfseries buildexamplepoissonsolver} lalu jalankan {\bfseries cmake} dengan flag spesial yaitu $\ast$$\ast$-\/\+D\+C\+M\+A\+K\+E\+\_\+\+P\+R\+E\+F\+I\+X\+\_\+\+P\+A\+T\+H$\ast$$\ast$ yang ditetapkan dengan absolute path tempat direktori \hyperlink{classPoissonSolver3DCylindricalGPU}{Poisson\+Solver3\+D\+Cylindrical\+G\+PU} dibangun.
\end{DoxyEnumerate}


\begin{DoxyCode}
$ mkdir buildexamplepoissonsolver
$ cd buildexamplepoissonsolver
$ cmake ../PoissonSolver3D/example/ -DCMAKE\_PREFIX\_PATH=/home/usertest/trypoissonsolver/buildpoissonsolver
...
-- Found CUDA: /apps/tools/cuda-9.1 (found version "9.1")
-- Building Poisson Solver  with CUDA support
-- Configuring done
-- Generating done
-- Build files have been written to: /home/usertest/trypoissonsolver/buildexamplepoissonsolver
\end{DoxyCode}

\begin{DoxyEnumerate}
\item Jalankan {\bfseries make} untuk membuat {\itshape executable file}
\end{DoxyEnumerate}


\begin{DoxyCode}
$ make

-- Building Poisson Solver  with CUDA support
-- Configuring done
-- Generating done
-- Build files have been written to: /home/usertest/trypoissonsolver/buildexamplepoissonsolver
[ 50%] Linking CXX executable poissonsolvergputest
[100%] Built target poissonsolvergputest
\end{DoxyCode}

\begin{DoxyEnumerate}
\item Hasil dari {\bfseries make} adalah program yang dapat dieksekusi yang berjalan di G\+PU card 
\begin{DoxyCode}
$ ./poissonsolvergputest
Poisson Solver 3D Cylindrical GPU test
Ukuran grid (r,phi,z) = (17,17,18)
Waktu komputasi:            0.55 s
Jumlah iterasi siklus multigrid:    5
Iterasi Error Convergen     Error Absolut
[0]:    8.362587e-01    2.430797e-03
[1]:    5.126660e-02    4.434380e-04
[2]:    9.490424e-03    1.892288e-04
[3]:    2.084086e-03    1.314385e-04
[4]:    1.141812e-03    1.217592e-04

Poisson Solver 3D Cylindrical GPU test
Ukuran grid (r,phi,z) = (33,33,36)
Waktu komputasi:            1.000000e-02 s
Jumlah iterasi siklus multigrid:    10
Iterasi Error Convergen     Error Absolut
[0]:    1.724319e-03    1.315177e-04
[1]:    8.652242e-05    1.266415e-04
[2]:    5.450314e-05    1.256187e-04
[3]:    5.081671e-05    1.249848e-04
[4]:    4.951065e-05    1.244460e-04
[5]:    4.904093e-05    1.241379e-04
[6]:    4.890266e-05    1.239894e-04
[7]:    4.885175e-05    1.239208e-04
[8]:    4.883604e-05    1.238959e-04
[9]:    4.883501e-05    1.238844e-04

Poisson Solver 3D Cylindrical GPU test
Ukuran grid (r,phi,z) = (65,65,72)
Waktu komputasi:            5.000000e-02 s
Jumlah iterasi siklus multigrid:    3
Iterasi Error Convergen     Error Absolut
[0]:    1.084726e-04    1.291784e-04
[1]:    5.680057e-06    1.283386e-04
[2]:    1.501479e-06    1.278558e-04

Poisson Solver 3D Cylindrical GPU test
Ukuran grid (r,phi,z) = (129,129,144)
Waktu komputasi:            1.200000e-01 s
Jumlah iterasi siklus multigrid:    3
Iterasi Error Convergen     Error Absolut
[0]:    1.373293e-05    1.302576e-04
[1]:    2.154564e-06    1.299626e-04
[2]:    1.048384e-06    1.297261e-04

Poisson Solver 3D Cylindrical GPU test
Ukuran grid (r,phi,z) = (257,257,288)
Waktu komputasi:            6.900000e-01 s
Jumlah iterasi siklus multigrid:    5
Iterasi Error Convergen     Error Absolut
[0]:    5.469890e-06    1.388922e-04
[1]:    1.783695e-06    1.384964e-04
[2]:    1.345343e-06    1.381199e-04
[3]:    1.156555e-06    1.377524e-04
[4]:    1.050234e-06    1.373983e-04
\end{DoxyCode}
 
\end{DoxyEnumerate}