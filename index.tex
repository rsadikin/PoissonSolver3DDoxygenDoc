Poisson\+Solver3D adalah pustaka yang dikembangkan untuk menyelesaikan persamaan Poisson 3 dimensi dalam sistem koordinat silinder. Persamaan Poisson secara umum berbentuk \$\$$^\wedge$\{2\}(r,,z) = (r,,z)\$\$ dengan diketahui nilai tepi pada potensial \$\$\{V\}\$\$ dan distribusi buatan \$\$\$\$.

Untuk menyelesaikan persamaan tersebut digunakan metode multigrid yang diimplementasikan pada akselerator G\+PU.

\subsection*{Pengembang}

Pustaka ini dikembangkan oleh Kelompok Penelitian Komputasi Kinerja Tinggi, Pusat Penelitian Informatika, Lembaga Ilmu Pengetahuan Indonesia. Tim pengembang terdiri dari\+:


\begin{DoxyItemize}
\item Rifki Sadikin (\href{mailto:rifki.sadikin@lipi.go.id}{\tt rifki.\+sadikin@lipi.\+go.\+id}), koordinator
\item I Wayan Aditya Swardiana (\href{mailto:i.wayan.aditya.swardiana@lipi.go.id}{\tt i.\+wayan.\+aditya.\+swardiana@lipi.\+go.\+id}), anggota
\item Taufiq Wirahman (\href{mailto:taufiq.wirahman@lipi.go.id}{\tt taufiq.\+wirahman@lipi.\+go.\+id}), anggota
\item Arnida L. Latifah (\href{mailto:arnida.lailatul.lattifah@lipigo.id}{\tt arnida.\+lailatul.\+lattifah@lipigo.\+id}), anggota \subsection*{Kontak}
\end{DoxyItemize}

Untuk pertanyaan, komentar atau diskusi dapat menghubungi tim pengembang di atas atau mengunjungi situs kami di\+: \href{http://grid.lipi.go.id}{\tt http\+://grid.\+lipi.\+go.\+id}

\subsection*{Dokumentasi}

Petunjuk penggunaan dapat dilihat pada berkas \hyperlink{PETUNJUKPENGGUNAAN_8md_source}{P\+E\+T\+U\+N\+J\+U\+K\+P\+E\+N\+G\+G\+U\+N\+A\+AN.md} 